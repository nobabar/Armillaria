%--------------------------------------------------------%
%	GLOSSAIRE
%--------------------------------------------------------%

\pagestyle{empty}

\section*{Glossaire}

\label{bib:biblioI}
Armillaria - Ce nom provient de l’anneau qu’ils portent sur leur pied. Cet anneau est appelé armille, du latin armilla signifiant bracelet.

\label{bib:biblioII}
Pourridié-agaric - Destruction de la cellulose et de la lignine des racines causant une pourriture blanche.

\label{bib:biblioIII}
Espèces européennes d'Armillaire - \textit{A. mellea}, \textit{A. borealis}, \textit{A. gallica}, \textit{A. cepistipes}, \textit{A. ostoyae} et \textit{A. tabescens}.

\label{bib:biblioIV}
Sporophore - C'est l'appareil reproducteur des champignons supérieurs, c'est ce qu'on appelle champignon dans le langage courant.

\label{bib:biblioV}
Thalles - Corps végétatif non différencié, chez les champignons il est formé d'hyphes, éléments filamenteux composant le mycélium et le sporophore.

\label{bib:biblioVI}
Rhizomorphes - Réseau d'hyphes permettant l'exploration, qu'elle soit souterraine ou sous l'écorce d'arbres.

\label{bib:biblioVII}
ITS - Internal Transcribed Spacer 1 - Cette séquence se trouve entre les gènes de deux sous-unités d'ARNr (18S et 5.8S). Elle est considérée comme le barcode des champignons car elle présente souvent plus de variations que les autres régions d'ARNr.

\label{bib:biblioVIII}
Complexe d'espèces - Groupement d'organismes dans lequel la séparation entre chaque espèce est peu évidente tant elles sont semblables.

\label{bib:biblioIX}
SNP - Single Nucleotide Polymorphism - variation génétique d'une seule paire de base à un endroit spécifique du génome et caractéristique d'un individu, d'une population ou d'une espèce.