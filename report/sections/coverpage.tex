%--------------------------------------------------------%
%	COVER PAGE
%--------------------------------------------------------%

\begin{titlepage}

  \newcommand{\HRule}{\rule{\linewidth}{0.5mm}} % Defines a new command for the horizontal lines, change thickness here


\begin{figure}[ht!]
\includesvg[width=0.3\linewidth]{figures/logo}
\end{figure}
\vspace{1.8 cm}

\begin{center}

%	HEADING SECTION
  \textsc{\LARGE Unité de Formation de Biologie}\\[0.5cm]
  \textsc{\Large Licence Sciences et Technologies}\\[0.2cm]
  \textsc{\large Mention Sciences de la Vie}\\
  \textsc{\large Parcour Sciences du Vivant}\\
  \textsc{\large L3}\\

%	TITLE SECTION
  \vspace{1 cm}
  \HRule \\[0.4cm]
  {\Huge \bfseries Rapport de Stage}\\[0.4cm]
  {\Huge \bfseries Identification des espèces européennes d'Armillaire à partir de variations nucléotidiques diagnostiques}\\
  \HRule \\[1cm]
 

\textbf{\LARGE Rousseau Baptiste}
\end{center}

\vspace{2.6cm}
\noindent\textsc{Dates du Stage : 17 mai - 25 juin}\\
\textsc{Durée du stage : 4 semaines}\\
\textsc{Maître de stage : Cyril Dutech}\\
\textsc{Nom et Adresse de la structure d’accueil :}
\setlength{\parindent}{5ex}
\par INRAE - Unité Expérimentale Forêt Pierroton
\par Site de Recherches Forêt Bois de Pierroton
\par 69 route d'Arcachon
\par 33612 Cestas Cedex - FRANCE

\vfill % Fill the rest of the page with whitespace
\newpage

%--------------------------------------------------------%
%	TABLE OF CONTENT
%--------------------------------------------------------%

\thispagestyle{empty}
\tableofcontents
\vspace{2cm}

%--------------------------------------------------------%
%	ABSTRACT
%--------------------------------------------------------%

\begin{abstract}
\noindent Le genre Armillaria, appartenant à l'embranchement des Basidiomycètes et de la famille des Physalacriaceae, est un complexe d'espèces aux comportements saprotrophes ou parasitaires. Il est donc étudié pour ses impacts sur la santé des forêt. Les espèces de ce complexe sont très proches et leur identification est un tâche délicate où les critères morphologiques et biologiques ont parfois du mal. Les critères phylogénétiques sont donc les plus courants, nécessitant cependant des données autant en quantité qu'en qualité  Nous développons ici une méthode d'identification basée sur des SNPs diagnostiques pour les espèces européennes. Les SNPs mis en évidence et les identifications concordent avec des identifications par critères phylogénétiques tout en étant plus simple d'utilisation est plus commode. Cette méthode montre 75\% d'efficacité dans les premiers essais, la proximité et le polymorphisme rendant la détermination et la détection de SNPs complexe. Ce qui montre, une fois de plus, la continuité de ce complexe d'espèce et la difficulté de sa  classification.
\end{abstract}

\vfill % Fill the rest of the page with whitespace

\end{titlepage}

%-----------------------------------------------------------------

\newpage