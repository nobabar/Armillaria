%--------------------------------------------------------%
%	CODE
%--------------------------------------------------------%

\subsection{Nettoyage des séquences}
\label{ann:annexeD1}
    \begin{tcolorbox}[colback=darkgray!5!white,colframe=black!75!black]
        \begin{minted}{Python}

# lecture du fichier contenant les séquences
ref_file_dir = r'C:\...\refseq.fasta'
with open(ref_file_dir, "r") as file:
    cont = file.read().split('>')[1:]

# les noms de séquences et les séquences sont séparées
name, seq = [], []
for x in cont:
    name.append(x.split('\n', 1)[0])
    seq.append(x.split('\n', 1)[1].replace('\n', ''))
species = list(i.rsplit(".", 1)[1] for i in name)

# on fait une liste loci contenant une liste de que chaque locus
# ainsi qu'une liste gaps contenant l'information d'un gap ou non pour chaque locus
loci = []
gaps = []
for i in range(len(seq[0])):
    locus = ''
    gp = ''
    for x in seq:
        locus += x[i]
        if x[i] == '-':
            gap = '-'
    loci.append(locus)
    gaps.append(gap)

# tant que le premier locus donné contient plus de 20% de gaps, l'enlever
while loci[0].count("-") > 0.2*len(cont):
    loci.pop(0)
# tant que le dernier locus donné contient plus de 20% de gaps, l'enlever
while loci[-1].count("-") > 0.2*len(cont):
    loci.pop(-1)
# pour éviter des problèmes d'alignement, enlever les sites contenant des gaps
while i < len(loci):
        if 0 < loci[i].count("-"): #< 0.2*len(cont):
            loci.pop(i)
        else:
            i += 1

        \end{minted}
    \end{tcolorbox}
\subsection{Création d'un fichier rassemblant les SNPs}
\label{ann:annexeD2}
    \begin{tcolorbox}[colback=darkgray!5!white,colframe=black!75!black]
        \begin{minted}{Python}
snp = {}
for i in range(len(loci)):
    # pas de SNP si le locus ne contient qu'un seul type de nucléotide
    if len(set(loci[i])) >= 2:
        snp_loc = []
        for j in range(len(loci[i])):
            snp_loc.append(list(loci[i])[j])
        tag = True
        for k in set(loci[i]):
            # on ne prend pas en compte le SNP si c'est un singleton
            if loci[i].count(k) == 1:
                tag = False
        # on ne le prend pas non plus en compte si c'est un indel
        if len(set(loci[i])) == 2 and "-" in set(loci[i]):
            tag = False
        # si toutes les conditions sont réunies alors
        # nous avons un locus susceptible de contenir un SNP
        if tag:
            snp[i+1] = snp_loc
    seq_final.append(loci[i])

d = {}
d["species"] = species
for i in snp.keys():
    d[i] = list(snp[i])
df = panda.DataFrame(data=d, columns=d.keys(), index=name)
df.to_csv(r'C\...\consensus_seq.csv')
        \end{minted}
    \end{tcolorbox}
\subsection{Données manquantes des séquences inconnues}
\label{ann:annexeD3}
    \begin{tcolorbox}
        \begin{minted}{Python}
i = 0
# parcours la liste des phred score
while i < len(seqQual):
    for j in range(len(seqQual[i])):
        # un qualité de 20 signifie une probabilité d'erreur de 1%
        if seqQual[i][j] < 20:
            if j == 0:
                seqStr[i] = seqStr[i][1:]
            else:
                seqStr[i] = seqStr[i][:j-1]+'N'+seqStr[i][j-1:]
    # si la séquence contient plus de 15% de N
    # elle considérée de trop mauvaise qualité
    if seqStr[i].count("N") > 0.15*len(seqStr[i]):
        seqName.pop(i)
        seqStr.pop(i)
        seqQual.pop(i)
    else:
        # les débuts et fins de séquence sont souvent de mauvaise qualité
        while seqStr[i][:5].count("N") > 4:
            seqStr[i] = seqStr[i][5:]
        while seqStr[i][-5:].count("N") > 4:
            seqStr[i] = seqStr[i][:-5]
        i += 1
        \end{minted}
    \end{tcolorbox}
\subsection{Filtration de l'alignement}
\label{ann:annexeD4}
    \begin{tcolorbox}
        \begin{minted}{Python}
# on rassemble les séquences références et les inconnues
with open(unaligned_dir, "a") as align:
    with open(reference_file_dir, "r") as ref:
        for line in ref:
            align.write(line)
    with open(file, "r") as unk:
        for line in unk:
            align.write(line)

# on les aligne
muscle_cline = MuscleCommandline(muscle_dir, input=unaligned_dir, out=aligned_dir)
stdout, stderr = muscle_cline()

i = 0
# on parcour une séquence de référence avant l'alignement
while i < len(ref_seq)-1:
    # si cette séquence de référence est différente pour un site 
    # après l'alignement, alors ce site est une insertion et on l'enlève
    if ref_seq[i] != aligned_ref[i]:
        aligned_ref.pop(i)
        loci.pop(i)
    else:
        i += 1
# et on enlève les insertions à la fin de la séquence si il y en a
loci = loci[:len(ref_seq)]

with open(aligned_dir.replace(".fasta", "_cleaned.fasta"), "w") as cln:
    for i in range(len(loci[0])):
        tag = ''
        for x in loci:
            tag = tag + x[i]
        # if name[i] not in ref_names:
        cln.write('>' + name[i] + '\n' + tag + '\n')
        \end{minted}
    \end{tcolorbox}
\subsection{Identification des isolats}
\label{ann:annexeD5}
    \begin{tcolorbox}
        \begin{minted}{Python}
def identifier_intra(sequence, snp_dict, group):
    id_dict = {}
    for specie in group:
        # on parcour la liste des SNPs
        for snp in snp_dict[specie].keys():
            nucl = sequence[snp-1]
            # si le nucléotide au site du SNP correspond
            if nucl in snp_dict[specie][snp]:
                id_dict[specie]["score"] = id_dict[specie]["score"] + 1
            # si le nucléotide est une donnée manquante
            if nucl == "N":
                id_dict[specie]["N"] = id_dict[specie]["N"] + 1
    return id_dict

def identifier_inter(sequence, snp_dict, group):
    id_dict = {}
    # on parcour la liste des SNPs
    for snp in snp_dict[group].keys():
        nucl = sequence[snp-1]
        # si le nucléotide au site du SNP correspond
        if nucl in snp_dict[group][snp]:
            id_dict["score"] = id_dict["score"] + 1
        # si le nucléotide est une donnée manquante
        if nucl == "N":
            id_dict["N"] = id_dict["N"] + 1
    return id_dict

def group_branch(sequence, snp_dict, phylo):
    specie_ind = {}
    # on prend séparement l'extragroupe
    if "out_group" in phylo.keys():
        out_group_specie = identifier_intra(sequence, snp_dict, phylo["out_group"])
        for specie in out_group_specie.keys():
            # si l'isolat correspond à au moins 90% des SNPs - ses données manquantes
            if out_group_specie[specie]["score"] >= int(0.9*len(snp_dict[specie])
                                                   -out_group_specie[specie]["N"]):
                specie_ind = {specie: [out_group_specie[specie]["score"],
                                       out_group_specie[specie]["N"]]}
    # si l'isolat n'a pas été identifié comme faisant parti de l'extragroupe
    if bool(specie_ind) == False:
        for groups in phylo["groups"]:
            group_inter = identifier_inter(sequence, snp_dict, groups)
            # on accepte un seul SNP non validé pour classer l'isolat
            if group_inter["score"] >= len(snp_dict[groups])-1:
                group_specie = identifier_intra(sequence, snp_dict,
                                                group["groups"][groups])
                for specie in group_specie.keys():
                    # on accepte un seul SNP non validé pour classer l'isolat
                    if group_specie[specie]["score"] >= len(snp_dict[specie])-1:
                        specie_ind[specie] = [group_specie[specie]["score"],
                                              group_specie[specie]["N"]]
    # si l'isolat n'a été identifié pour aucune espèce
    if bool(specie_ind) == False:
        specie_ind = "undefined"
    return specie_ind
        \end{minted}
    \end{tcolorbox}
\subsection{Récupération de données GenBank}
\label{ann:annexeD6}
    \begin{tcolorbox}
        \begin{minted}{Python}
from Bio import Entrez
# Répeter l'opération pour chaque espèce de notre liste
for specie in species:
    handle_1 = Entrez.esearch(db="nucleotide",
                            term="armillaria"+specie+"internal transcribed spacer",
                            retmax=40,
                            retmode="fasta")
    # liste des numéro d'accès des séquences à récuperer
    id_list = ",".join(Entrez.read(handle_1)["IdList"])
    handle_1.close()
    
    handle_2 = Entrez.efetch(db = 'nucleotide', id = id_list,
                            rettype = 'fasta', retmod = 'text')
        \end{minted}
    \end{tcolorbox}
